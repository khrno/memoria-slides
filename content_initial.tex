\titlepageframe

\begin{tframe}{Introducción}
	\begin{adv}
		\item<1-> Fuerte \highlightbf{crecimiento} de generadores de información.
		\item<1-> \highlightbf{Facilidad} de acceso a los datos.
		\item<1-> Generación de \highlightbf{comunidades} en torno a tópicos.
	\end{adv}
	\begin{disadv}
		\item<2-> Capacidad \highlightbf{limitada} de procesamiento.
		\item<2-> Búsquedas \highlightbf{ineficientes} de datos.
		\item<2-> \highlightbf{Desinformación} producto de resultados errados.
	\end{disadv}
	\pause
\end{tframe}

\begin{frame}[c]{Introducción}
	\begin{block}{Problema de Exploración}
		¿Cuál data es \highlightbf{relevante} para las necesidades requeridas?
	\end{block}\pause
	\begin{block}{Problema de Explotación}
		¿Cómo utilizar esa información \highlightbf{eficientemente}?	
	\end{block}
\end{frame}

\begin{frame}[c]{Problema}
	\begin{columns}[onlytextwidth]
		\begin{column}{0.5\textwidth}
			\centering
    		\includegraphics<1>[width=0.5\textwidth]{images/document-icon.png}
  		\end{column}
  		\begin{column}{0.45\textwidth}
  			\begin{block}{Con \textbf{1} documento$\dots$}
				Ningún problema señalado ocurre.
  			\end{block}
		\end{column}
	\end{columns}
\end{frame}

\begin{frame}[c]{Problema}
	\begin{columns}[onlytextwidth]
		\begin{column}{0.5\textwidth}
    		\includegraphics<1>[width=0.3\textwidth]{images/document-icon.png}
    		\includegraphics<1>[width=0.3\textwidth]{images/document-icon.png}
    		\includegraphics<1>[width=0.3\textwidth]{images/document-icon.png}
    		\\
    		\includegraphics<1>[width=0.3\textwidth]{images/document-icon.png}
    		\includegraphics<1>[width=0.3\textwidth]{images/document-icon.png}
    		\includegraphics<1>[width=0.3\textwidth]{images/document-icon.png}
    		\\
    		\includegraphics<1>[width=0.3\textwidth]{images/document-icon.png}
    		\includegraphics<1>[width=0.3\textwidth]{images/document-icon.png}
    		\includegraphics<1>[width=0.3\textwidth]{images/document-icon.png}
  		\end{column}
  		\begin{column}{0.45\textwidth}
  			\begin{block}{Con \textbf{9} documentos$\dots$}
	  			Todavía ningún problema señalado ocurre.
  			\end{block}
		\end{column}
	\end{columns}
\end{frame}
\begin{frame}[c]{Problema}
	\begin{columns}[onlytextwidth]
		\begin{column}{0.5\textwidth}
    		\includegraphics<1>[width=0.2\textwidth]{images/document-icon.png}
    		\includegraphics<1>[width=0.2\textwidth]{images/document-icon.png}
    		\includegraphics<1>[width=0.2\textwidth]{images/document-icon.png}
    		\includegraphics<1>[width=0.2\textwidth]{images/document-icon.png}
    		\includegraphics<1>[width=0.2\textwidth]{images/document-icon.png}
    		\\
    		\includegraphics<1>[width=0.2\textwidth]{images/document-icon.png}
    		\includegraphics<1>[width=0.2\textwidth]{images/document-icon.png}
    		\includegraphics<1>[width=0.2\textwidth]{images/document-icon.png}
    		\includegraphics<1>[width=0.2\textwidth]{images/document-icon.png}
    		\includegraphics<1>[width=0.2\textwidth]{images/document-icon.png}
    		\\
    		\includegraphics<1>[width=0.2\textwidth]{images/document-icon.png}
    		\includegraphics<1>[width=0.2\textwidth]{images/document-icon.png}
    		\includegraphics<1>[width=0.2\textwidth]{images/document-icon.png}
    		\includegraphics<1>[width=0.2\textwidth]{images/document-icon.png}
    		\includegraphics<1>[width=0.2\textwidth]{images/document-icon.png}
    		\\
    		\includegraphics<1>[width=0.2\textwidth]{images/document-icon.png}
    		\includegraphics<1>[width=0.2\textwidth]{images/document-icon.png}
    		\includegraphics<1>[width=0.2\textwidth]{images/document-icon.png}
    		\includegraphics<1>[width=0.2\textwidth]{images/document-icon.png}
    		\includegraphics<1>[width=0.2\textwidth]{images/document-icon.png}
    		\\
    		\includegraphics<1>[width=0.2\textwidth]{images/document-icon.png}
    		\includegraphics<1>[width=0.2\textwidth]{images/document-icon.png}
    		\includegraphics<1>[width=0.2\textwidth]{images/document-icon.png}
    		\includegraphics<1>[width=0.2\textwidth]{images/document-icon.png}
    		\includegraphics<1>[width=0.2\textwidth]{images/document-icon.png}
  		\end{column}
  		\begin{column}{0.45\textwidth}
  			\begin{block}{Con \textbf{25} documentos$\dots$}
	  			Quizás tomé un poco de tiempo, pero ningún problema señalado ocurre.
  			\end{block}
		\end{column}
	\end{columns}
\end{frame}
\begin{frame}[c]{Problema}
\centering
\includegraphics<1>[width=0.05\textwidth]{images/document-collection-icon.png}
\includegraphics<1>[width=0.05\textwidth]{images/document-collection-icon.png}
\includegraphics<1>[width=0.05\textwidth]{images/document-collection-icon.png}
\includegraphics<1>[width=0.05\textwidth]{images/document-collection-icon.png}
\includegraphics<1>[width=0.05\textwidth]{images/document-collection-icon.png}
\includegraphics<1>[width=0.05\textwidth]{images/document-collection-icon.png}
\includegraphics<1>[width=0.05\textwidth]{images/document-collection-icon.png}
\includegraphics<1>[width=0.05\textwidth]{images/document-collection-icon.png}
\includegraphics<1>[width=0.05\textwidth]{images/document-collection-icon.png}
\includegraphics<1>[width=0.05\textwidth]{images/document-collection-icon.png}
\includegraphics<1>[width=0.05\textwidth]{images/document-collection-icon.png}
\includegraphics<1>[width=0.05\textwidth]{images/document-collection-icon.png}
\includegraphics<1>[width=0.05\textwidth]{images/document-collection-icon.png}
\includegraphics<1>[width=0.05\textwidth]{images/document-collection-icon.png}
\includegraphics<1>[width=0.05\textwidth]{images/document-collection-icon.png}
\includegraphics<1>[width=0.05\textwidth]{images/document-collection-icon.png}
\includegraphics<1>[width=0.05\textwidth]{images/document-collection-icon.png}
\includegraphics<1>[width=0.05\textwidth]{images/document-collection-icon.png}
\includegraphics<1>[width=0.05\textwidth]{images/document-collection-icon.png}
\includegraphics<1>[width=0.05\textwidth]{images/document-collection-icon.png}
\\
\includegraphics<1>[width=0.05\textwidth]{images/document-collection-icon.png}
\includegraphics<1>[width=0.05\textwidth]{images/document-collection-icon.png}
\includegraphics<1>[width=0.05\textwidth]{images/document-collection-icon.png}
\includegraphics<1>[width=0.05\textwidth]{images/document-collection-icon.png}
\includegraphics<1>[width=0.05\textwidth]{images/document-collection-icon.png}
\includegraphics<1>[width=0.05\textwidth]{images/document-collection-icon.png}
\includegraphics<1>[width=0.05\textwidth]{images/document-collection-icon.png}
\includegraphics<1>[width=0.05\textwidth]{images/document-collection-icon.png}
\includegraphics<1>[width=0.05\textwidth]{images/document-collection-icon.png}
\includegraphics<1>[width=0.05\textwidth]{images/document-collection-icon.png}
\includegraphics<1>[width=0.05\textwidth]{images/document-collection-icon.png}
\includegraphics<1>[width=0.05\textwidth]{images/document-collection-icon.png}
\includegraphics<1>[width=0.05\textwidth]{images/document-collection-icon.png}
\includegraphics<1>[width=0.05\textwidth]{images/document-collection-icon.png}
\includegraphics<1>[width=0.05\textwidth]{images/document-collection-icon.png}
\includegraphics<1>[width=0.05\textwidth]{images/document-collection-icon.png}
\includegraphics<1>[width=0.05\textwidth]{images/document-collection-icon.png}
\includegraphics<1>[width=0.05\textwidth]{images/document-collection-icon.png}
\includegraphics<1>[width=0.05\textwidth]{images/document-collection-icon.png}
\includegraphics<1>[width=0.05\textwidth]{images/document-collection-icon.png}
\\
\includegraphics<1>[width=0.05\textwidth]{images/document-collection-icon.png}
\includegraphics<1>[width=0.05\textwidth]{images/document-collection-icon.png}
\includegraphics<1>[width=0.05\textwidth]{images/document-collection-icon.png}
\includegraphics<1>[width=0.05\textwidth]{images/document-collection-icon.png}
\includegraphics<1>[width=0.05\textwidth]{images/document-collection-icon.png}
\includegraphics<1>[width=0.05\textwidth]{images/document-collection-icon.png}
\includegraphics<1>[width=0.05\textwidth]{images/document-collection-icon.png}
\includegraphics<1>[width=0.05\textwidth]{images/document-collection-icon.png}
\includegraphics<1>[width=0.05\textwidth]{images/document-collection-icon.png}
\includegraphics<1>[width=0.05\textwidth]{images/document-collection-icon.png}
\includegraphics<1>[width=0.05\textwidth]{images/document-collection-icon.png}
\includegraphics<1>[width=0.05\textwidth]{images/document-collection-icon.png}
\includegraphics<1>[width=0.05\textwidth]{images/document-collection-icon.png}
\includegraphics<1>[width=0.05\textwidth]{images/document-collection-icon.png}
\includegraphics<1>[width=0.05\textwidth]{images/document-collection-icon.png}
\includegraphics<1>[width=0.05\textwidth]{images/document-collection-icon.png}
\includegraphics<1>[width=0.05\textwidth]{images/document-collection-icon.png}
\includegraphics<1>[width=0.05\textwidth]{images/document-collection-icon.png}
\includegraphics<1>[width=0.05\textwidth]{images/document-collection-icon.png}
\includegraphics<1>[width=0.05\textwidth]{images/document-collection-icon.png}
\\
\includegraphics<1>[width=0.05\textwidth]{images/document-collection-icon.png}
\includegraphics<1>[width=0.05\textwidth]{images/document-collection-icon.png}
\includegraphics<1>[width=0.05\textwidth]{images/document-collection-icon.png}
\includegraphics<1>[width=0.05\textwidth]{images/document-collection-icon.png}
\includegraphics<1>[width=0.05\textwidth]{images/document-collection-icon.png}
\includegraphics<1>[width=0.05\textwidth]{images/document-collection-icon.png}
\includegraphics<1>[width=0.05\textwidth]{images/document-collection-icon.png}
\includegraphics<1>[width=0.05\textwidth]{images/document-collection-icon.png}
\includegraphics<1>[width=0.05\textwidth]{images/document-collection-icon.png}
\includegraphics<1>[width=0.05\textwidth]{images/document-collection-icon.png}
\includegraphics<1>[width=0.05\textwidth]{images/document-collection-icon.png}
\includegraphics<1>[width=0.05\textwidth]{images/document-collection-icon.png}
\includegraphics<1>[width=0.05\textwidth]{images/document-collection-icon.png}
\includegraphics<1>[width=0.05\textwidth]{images/document-collection-icon.png}
\includegraphics<1>[width=0.05\textwidth]{images/document-collection-icon.png}
\includegraphics<1>[width=0.05\textwidth]{images/document-collection-icon.png}
\includegraphics<1>[width=0.05\textwidth]{images/document-collection-icon.png}
\includegraphics<1>[width=0.05\textwidth]{images/document-collection-icon.png}
\includegraphics<1>[width=0.05\textwidth]{images/document-collection-icon.png}
\includegraphics<1>[width=0.05\textwidth]{images/document-collection-icon.png}
\\
\includegraphics<1>[width=0.1\textwidth]{images/infinite-icon-red.png}
\\
\includegraphics<1>[width=0.05\textwidth]{images/document-collection-icon.png}
\includegraphics<1>[width=0.05\textwidth]{images/document-collection-icon.png}
\includegraphics<1>[width=0.05\textwidth]{images/document-collection-icon.png}
\includegraphics<1>[width=0.05\textwidth]{images/document-collection-icon.png}
\includegraphics<1>[width=0.05\textwidth]{images/document-collection-icon.png}
\includegraphics<1>[width=0.05\textwidth]{images/document-collection-icon.png}
\includegraphics<1>[width=0.05\textwidth]{images/document-collection-icon.png}
\includegraphics<1>[width=0.05\textwidth]{images/document-collection-icon.png}
\includegraphics<1>[width=0.05\textwidth]{images/document-collection-icon.png}
\includegraphics<1>[width=0.05\textwidth]{images/document-collection-icon.png}
\includegraphics<1>[width=0.05\textwidth]{images/document-collection-icon.png}
\includegraphics<1>[width=0.05\textwidth]{images/document-collection-icon.png}
\includegraphics<1>[width=0.05\textwidth]{images/document-collection-icon.png}
\includegraphics<1>[width=0.05\textwidth]{images/document-collection-icon.png}
\includegraphics<1>[width=0.05\textwidth]{images/document-collection-icon.png}
\includegraphics<1>[width=0.05\textwidth]{images/document-collection-icon.png}
\includegraphics<1>[width=0.05\textwidth]{images/document-collection-icon.png}
\includegraphics<1>[width=0.05\textwidth]{images/document-collection-icon.png}
\includegraphics<1>[width=0.05\textwidth]{images/document-collection-icon.png}
\includegraphics<1>[width=0.05\textwidth]{images/document-collection-icon.png}
\\
\includegraphics<1>[width=0.05\textwidth]{images/document-collection-icon.png}
\includegraphics<1>[width=0.05\textwidth]{images/document-collection-icon.png}
\includegraphics<1>[width=0.05\textwidth]{images/document-collection-icon.png}
\includegraphics<1>[width=0.05\textwidth]{images/document-collection-icon.png}
\includegraphics<1>[width=0.05\textwidth]{images/document-collection-icon.png}
\includegraphics<1>[width=0.05\textwidth]{images/document-collection-icon.png}
\includegraphics<1>[width=0.05\textwidth]{images/document-collection-icon.png}
\includegraphics<1>[width=0.05\textwidth]{images/document-collection-icon.png}
\includegraphics<1>[width=0.05\textwidth]{images/document-collection-icon.png}
\includegraphics<1>[width=0.05\textwidth]{images/document-collection-icon.png}
\includegraphics<1>[width=0.05\textwidth]{images/document-collection-icon.png}
\includegraphics<1>[width=0.05\textwidth]{images/document-collection-icon.png}
\includegraphics<1>[width=0.05\textwidth]{images/document-collection-icon.png}
\includegraphics<1>[width=0.05\textwidth]{images/document-collection-icon.png}
\includegraphics<1>[width=0.05\textwidth]{images/document-collection-icon.png}
\includegraphics<1>[width=0.05\textwidth]{images/document-collection-icon.png}
\includegraphics<1>[width=0.05\textwidth]{images/document-collection-icon.png}
\includegraphics<1>[width=0.05\textwidth]{images/document-collection-icon.png}
\includegraphics<1>[width=0.05\textwidth]{images/document-collection-icon.png}
\includegraphics<1>[width=0.05\textwidth]{images/document-collection-icon.png}
\\
\includegraphics<1>[width=0.05\textwidth]{images/document-collection-icon.png}
\includegraphics<1>[width=0.05\textwidth]{images/document-collection-icon.png}
\includegraphics<1>[width=0.05\textwidth]{images/document-collection-icon.png}
\includegraphics<1>[width=0.05\textwidth]{images/document-collection-icon.png}
\includegraphics<1>[width=0.05\textwidth]{images/document-collection-icon.png}
\includegraphics<1>[width=0.05\textwidth]{images/document-collection-icon.png}
\includegraphics<1>[width=0.05\textwidth]{images/document-collection-icon.png}
\includegraphics<1>[width=0.05\textwidth]{images/document-collection-icon.png}
\includegraphics<1>[width=0.05\textwidth]{images/document-collection-icon.png}
\includegraphics<1>[width=0.05\textwidth]{images/document-collection-icon.png}
\includegraphics<1>[width=0.05\textwidth]{images/document-collection-icon.png}
\includegraphics<1>[width=0.05\textwidth]{images/document-collection-icon.png}
\includegraphics<1>[width=0.05\textwidth]{images/document-collection-icon.png}
\includegraphics<1>[width=0.05\textwidth]{images/document-collection-icon.png}
\includegraphics<1>[width=0.05\textwidth]{images/document-collection-icon.png}
\includegraphics<1>[width=0.05\textwidth]{images/document-collection-icon.png}
\includegraphics<1>[width=0.05\textwidth]{images/document-collection-icon.png}
\includegraphics<1>[width=0.05\textwidth]{images/document-collection-icon.png}
\includegraphics<1>[width=0.05\textwidth]{images/document-collection-icon.png}
\includegraphics<1>[width=0.05\textwidth]{images/document-collection-icon.png}
\\
\includegraphics<1>[width=0.05\textwidth]{images/document-collection-icon.png}
\includegraphics<1>[width=0.05\textwidth]{images/document-collection-icon.png}
\includegraphics<1>[width=0.05\textwidth]{images/document-collection-icon.png}
\includegraphics<1>[width=0.05\textwidth]{images/document-collection-icon.png}
\includegraphics<1>[width=0.05\textwidth]{images/document-collection-icon.png}
\includegraphics<1>[width=0.05\textwidth]{images/document-collection-icon.png}
\includegraphics<1>[width=0.05\textwidth]{images/document-collection-icon.png}
\includegraphics<1>[width=0.05\textwidth]{images/document-collection-icon.png}
\includegraphics<1>[width=0.05\textwidth]{images/document-collection-icon.png}
\includegraphics<1>[width=0.05\textwidth]{images/document-collection-icon.png}
\includegraphics<1>[width=0.05\textwidth]{images/document-collection-icon.png}
\includegraphics<1>[width=0.05\textwidth]{images/document-collection-icon.png}
\includegraphics<1>[width=0.05\textwidth]{images/document-collection-icon.png}
\includegraphics<1>[width=0.05\textwidth]{images/document-collection-icon.png}
\includegraphics<1>[width=0.05\textwidth]{images/document-collection-icon.png}
\includegraphics<1>[width=0.05\textwidth]{images/document-collection-icon.png}
\includegraphics<1>[width=0.05\textwidth]{images/document-collection-icon.png}
\includegraphics<1>[width=0.05\textwidth]{images/document-collection-icon.png}
\includegraphics<1>[width=0.05\textwidth]{images/document-collection-icon.png}
\includegraphics<1>[width=0.05\textwidth]{images/document-collection-icon.png}
\end{frame}

\begin{tframe}{Problema}
	Con infinitos documentos$\ldots$
	\begin{disadv}
		\item ¿Cómo puedo detectar los documentos relevantes a mis necesidades? (\emph{Information Retrieval}).
		\item ¿De qué habla esta gran colección de documentos? (\emph{Natural Language Processing}).
		\item Detección de \emph{milestones} dentro de la colección.
		\item Navegación a través de los contenidos de la colección y no a través de los documentos que la componen (\emph{Linked Data}).
		\item Generar una estructura que sea capaz de almacenar estos documentos de forma lógica y eficiente (\emph{Information Retrieval}).
		\item Automatizar la tarea de mantener la colección de documentos vigente.
	\end{disadv}
\end{tframe}

\begin{tframe}{Problema}
	\begin{block}{Comunidad Científica}
		\begin{itemize}
			\item ¿Cómo puedo obtener aquellos artículos relacionados con la investigación que estoy haciendo o que deseo realizar?
			\item ¿Cómo puedo detectar quiénes han sido los precursores de las ideas detrás de las técnicas?
			\item Actualmente, resolver estás preguntas \highlightbf{consume gran parte} del tiempo de un investigador científico.
		\end{itemize}
	\end{block}
\end{tframe}

\begin{frame}[c]{Solución}
	\begin{block}{Procesamiento de la información}
		\begin{itemize}
			\item La colección documental utilizada es la base de datos bibliográfica \highlightbf{DBLP}.
			\item Procesar enormes colecciones documentales a través de técnicas de \emph{Information Retrieval} como lo es \emph{Formal Concept Analysis} y dentro del ámbito de \emph{Topic Modeling} se utilizará la técnica \emph{Latent Dirichlet Allocation}.
		\end{itemize}
	\end{block}
\end{frame}

\begin{frame}[c]{Solución}
	\begin{block}{Resumen Interactivo}
		Generar una visualización que permita
		\begin{itemize}
			\item Navegar a través de los distintos ``conceptos formales''.
			\item Analizar la distribución de ``tópicos emergentes''.		\end{itemize}
	\end{block}
\end{frame}

\begin{frame}[c]{Marco Teórico - \emph{DBLP}}
	\centering
   	\includegraphics<1>[width=0.5\textwidth]{images/dblp-logo.png}
	\begin{itemize}
		\item Plataforma Web alojada en Alemania que contiene artículos científicos relacionados con ciencias de la computación.
		\item En los años 80's fue una base de datos pequeñas relacionada a través de programación lógica.
		\item Contiene artículos de las revistas \emph{VLDB}, \emph{IEEE}, \emph{ACM}, además de distintas conferencias. 
	\end{itemize}
\end{frame}

\begin{frame}[c]{Marco Teórico - \emph{FCA}}
	\begin{itemize}
		\item Método de análisis de datos.
		\item Analiza la información que describe la relación entre un particular conjunto de objetos y atributos.
		\item Produce dos salidas
		\begin{itemize}
			\item \emph{Concept Lattice}
			\item Implicaciones de atributos
		\end{itemize}
	\end{itemize}
\end{frame}

\begin{frame}[c]{Marco Teórico - \emph{FCA}}
	\begin{columns}[onlytextwidth]
		\begin{column}{0.5\textwidth}
			\begin{table}[h!]
				\centering
				\begin{tabular}{|c|cccc|}
					\hline
					$I$		& $y_1$		& $y_2$ 	& $y_3$ 	& $y_4$ 	\\ \hline
					$x_1$	& $\times$	& $\times$	& $\times$ 	& $\times$ 	\\
					$x_2$	& $\times$	& 			& $\times$ 	& $\times$ 	\\
					$x_3$	& 			& $\times$	& $\times$ 	& $\times$ 	\\
					$x_4$	& 			& $\times$	& $\times$ 	& $\times$ 	\\
					$x_5$	& $\times$	& 			& 		 	& 		 	\\
					\hline
				\end{tabular}
				\label{tbl:cross_table}
				\caption{Contexto Formal}
			\end{table}
		\end{column}
		\vrule{}
		\begin{column}{0.5\textwidth}
					\begin{table}[h!]
				\centering
				\begin{tabular}{|cccc|}
					\hline
					$\times$	& $\times$  & \highlightbf{$\times$}  & \highlightbf{$\times$}    \\
					$\times$	&           & \highlightbf{$\times$}  & \highlightbf{$\times$}    \\
								& $\times$  & \highlightbf{$\times$}  & \highlightbf{$\times$}    \\
								& $\times$  & \highlightbf{$\times$}  & \highlightbf{$\times$}    \\
					$\times$	&           &                     		&                     		\\
					\hline
				\end{tabular}
				\begin{tabular}{|cccc|}
					\hline
					$\times$	& \highlightbf{$\times$}  & \highlightbf{$\times$}  & \highlightbf{$\times$}  \\
					$\times$	&                         & $\times$                & $\times$                \\
								& \highlightbf{$\times$}  & \highlightbf{$\times$}  & \highlightbf{$\times$}  \\
								& \highlightbf{$\times$}  & \highlightbf{$\times$}  & \highlightbf{$\times$}  \\
					$\times$	&                         &                         &                         \\
					\hline
				\end{tabular}
			\end{table}
		\end{column}
	\end{columns}
\end{frame}

\begin{frame}[c]{Marco Teórico - \emph{FCA}}
	\begin{columns}[onlytextwidth]
		\begin{column}{0.5\textwidth}
			\begin{table}[h!]
				\centering
				\begin{tabular}{|c|cccc|}
					\hline
					$I$		& $y_1$		& $y_2$ 	& $y_3$ 	& $y_4$ 	\\ \hline
					$x_1$	& $\times$	& $\times$	& $\times$ 	& $\times$ 	\\
					$x_2$	& $\times$	& 			& $\times$ 	& $\times$ 	\\
					$x_3$	& 			& $\times$	& $\times$ 	& $\times$ 	\\
					$x_4$	& 			& $\times$	& $\times$ 	& $\times$ 	\\
					$x_5$	& $\times$	& 			& 		 	& 		 	\\
					\hline
				\end{tabular}
				\label{tbl:cross_table}
				\caption{Contexto Formal}
			\end{table}
		\end{column}
		\vrule{}
		\begin{column}{0.5\textwidth}
			\begin{table}[h!]
				\centering
				\begin{tabular}{|cccc|}
					\hline
					\highlightbf{$\times$}  & $\times$      & \highlightbf{$\times$}  & \highlightbf{$\times$}  \\
					\highlightbf{$\times$}  &               & \highlightbf{$\times$}  & \highlightbf{$\times$}  \\
											& $\times$      & $\times$                & $\times$                \\
											& $\times$      & $\times$                & $\times$                \\
					$\times$                &               &                         &                         \\
			        \hline
			    \end{tabular}
			    \begin{tabular}{|cccc|}
			    	\hline
			    	\highlightbf{$\times$}  & $\times$  & $\times$  & $\times$  \\
			    	\highlightbf{$\times$}  &           & $\times$  & $\times$  \\
											& $\times$  & $\times$  & $\times$  \\
											& $\times$  & $\times$  & $\times$  \\
					\highlightbf{$\times$}  &           &           &           \\
					\hline
				\end{tabular}
			\end{table}
		\end{column}
	\end{columns}
\end{frame}

\begin{frame}[c]{Marco Teórico - \emph{FCA}}
	\centering
	\includegraphics<1>[width=0.5\textwidth]{images/first-lattice.png}
\end{frame}

\begin{tframe}{Marco Teórico - \emph{FCA}}
	\begin{block}{Soporte Mínimo}
		El \highlightbf{\emph{support}} de un concepto formal dado por $\langle A,B \rangle$, donde $A \subseteq X$ y $B \subseteq Y$ está definido por:
 		\begin{equation*}
			\text{\emph{supp}}(\langle A,B \rangle) = \frac{|A|}{|X|}
		\end{equation*}
	\end{block}
	\begin{block}{\emph{Frequent Concept}}
		Dado un umbral $\text{\emph{minsupp}} \in [0,1]$, entonces el concepto $\langle A,B \rangle$ es llamado \emph{Frequent Concept} si y sólo si $\text{\emph{supp}}(\langle A,B \rangle) \ge \text{\emph{minsupp}}$.
	\end{block}
\end{tframe}

\begin{frame}[c]{Marco Teórico - \emph{FCA}}
	\begin{block}{Iceberg Lattice}
		 Un \highlightbf{\emph{Iceberg Lattice}} es el conjunto de todos los \emph{Frequent Concepts} dado un \emph{minsupp}
	\end{block}	
\end{frame}

\begin{frame}[c]{Marco Teórico - \emph{LDA}}
	\begin{itemize}
		\item Modelo perteneciente al área \emph{Topic Modeling}
		\item Busca descubrir tópicos a partir de una gran colección de documentos.
		\item \highlightbf{LDA} asume que:
		\begin{itemize}
			\item Un documento $D$ habla sobre un conjunto limitado de \highlightbf{Tópicos}.
			\item Un \emph{tópico} se compone a través de un \highlightbf{vocabulario fijo}.
		\end{itemize}
		\item LDA es un proceso generativo que utiliza técnicas de \highlightbf{Inferencia Estadística} para detectar los tópicos de una gran cantidad de datos.
	\end{itemize}
\end{frame}

\begin{frame}[c]{Marco Teórico - \emph{LDA}}
	\centering
	\includegraphics<1>[width=1\textwidth]{images/lda-process.png}
\end{frame}


\begin{tframe}{Marco Teórico - \emph{D3.js}}
	\includegraphics<1>[width=1\textwidth]{images/D3js-Logo.png}
	\begin{itemize}
		\item Librería \emph{Open Source} de \emph{Javascript}
		\item Ideada para manipular documentos basados en información.
		\item Componente fuerte en manipulación del \emph{DOM} de un sitio web.
		\item Ideal para generar \highlightbf{herramientas interactivas} 
	\end{itemize}
\end{tframe}


\begin{frame}[c]{Resultados}
	\centering
	\large{\highlightbf{Visión Global del Lattice}}
	\\
	\includegraphics<1>[width=0.7\textwidth]{images/results_1.png}
\end{frame}

\begin{frame}[c]{Resultados}
	\centering
	\large{\highlightbf{Selector de nivel}}
	\\
	\includegraphics<1>[width=0.7\textwidth]{images/results_3.png}
\end{frame}

\begin{frame}[c]{Resultados}
	\centering
	\large{\highlightbf{\emph{Preview} del Concepto Formal}}
	\\
	\includegraphics<1>[width=0.7\textwidth]{images/results_4.png}
\end{frame}

\begin{frame}[c]{Resultados}
	\centering
	\large{\highlightbf{Detalle del Concepto Formal}}
	\\
	\includegraphics<1>[width=0.7\textwidth]{images/results_6.png}
\end{frame}

\begin{frame}[c]{Resultados}
	\centering
	\large{\highlightbf{Detalle del Concepto Formal}}
	\\
	\includegraphics<1>[width=0.4\textwidth]{images/results_7.png}
\end{frame}

\begin{frame}[c]{Resultados}
	\centering
	\large{\highlightbf{Detalle del Concepto Formal}}
	\\
	\includegraphics<1>[width=0.4\textwidth]{images/results_8.png}
\end{frame}

\begin{frame}[c]{Resultados}
	\centering
	\large{\highlightbf{Detalle del Concepto Formal}}
	\\
	\includegraphics<1>[width=1\textwidth]{images/results_9.png}
\end{frame}

\begin{tframe}{Resultados}
	\centering
	\large{\highlightbf{Selector de parámetro \emph{minsupp}}}
	\\
	\includegraphics<1>[width=1\textwidth]{images/results_10.png}
\end{tframe}

\begin{tframe}{Conclusiones}	
	\begin{enumerate}
		\item Problema de la \highlightbf{dimensionalidad}.
		\item Fuerte relación entre las técnicas utilizadas.
		\item Librerías gráficas flexibles entregando al usuario una gran capacidad de interactuar / navegar.
		\item Resumen visual, interactivo y navegable de una gran colección de datos.
	\end{enumerate}
\end{tframe}

\begin{tframe}{Trabajo Futuro}
	\begin{enumerate}
		\item Extender este trabajo para analizar las \highlightbf{redes sociales} que forman los autores / consumidores en torno a los tópicos descubiertos.
		\item Extender el análisis para incluir un \highlightbf{análisis de sentimientos}.
		\item Crear componente para la tarea de la \highlightbf{recolección} de datos.
		\item Monitoreo de redes sociales.
		\item Alertas tempranas de eventos específicos.
		\item Muchas otras aplicaciones$\dots$
	\end{enumerate}
\end{tframe}



